\documentclass[a4paper,hidelinks,8pt]{article}


\usepackage{titlesec}
\usepackage{titling}
\usepackage[margin=1.3cm]{geometry}
\usepackage{parskip}
\usepackage[hidelinks]{hyperref}
\usepackage{helvet}
\pagenumbering{gobble}
\titleformat{\section}
{\huge\bfseries}
{\hspace{-0.6cm}}
{0em}
{}[\titlerule]

\titleformat{\subsection}[runin]
{\bfseries\large}
{$\bullet$}
{0.4em}
{}[ ---]

\titlespacing{\section}
{0em}{0em}{0.4em}


\titlespacing{\subsection}
{-1em}{0.4em}{1em}

\titlespacing{\subsubsection}
{0em}{0em}{0em}

\titleformat{\subsubsection}
{\bfseries}
{}
{0em}
{}

\begin{document}
\title{R\'esum\'e}
\author{Iurea Marius}
\renewcommand{\maketitle}
{
\begin{center}
        {\huge\bfseries
        \thetitle

        \theauthor}

        marius.iurea@yahoo.com --- 0741055112 --- \href{https://www.linkedin.com/in/marius-iurea-49b7021b5/}{LinkedIn}

        \end{center}
}

\maketitle

\section{Personal Summary}
A self-motivated individual and passionate software engineer, eager to build products that solve user problems and to take on new challenges.
Team player with good communication skills, adaptable to any work environment. \href{https://github.com/IureaMarius/Resume/blob/master/Resume.pdf}{(Up-to-date version of the r\'esum\'e)}
\section{Technical Skills}
\subsection{Languages}
C\#, JavaScript, C, C++, Python, HTML
\subsection{Frameworks/Libraries}
 ASP.NET, Node.js, Express.js, JQuery, Django
\subsection{Tools}
Git, Unity, Visual Studio, Vim
\section{Experience}

\subsection{IOmundo}Software Engineer | 01/10/2020 $\rightarrow$ current
\subsubsection{Technologies used: C\#, Angular}
\subsubsection{Responsibilities:}
-learned Angular over the course of 2 weeks while discussing the project requirements and working on the basic HTML/CSS structure of the application

-created a bridge to convert our existing XML based API into a JSON based API

-added endpoints to expose our backend Umbraco CMS

-fixed various bugs in our backend

-added internationalization support

-optimized the front end rendering of search results by adding virtual scrolling

-found an issue in our content retrieval Umbraco controller that caused a big part of the content stored in Umbraco to be retrieved from the database, instead of the more efficient XML cache. After adding an Lucene search index, and making sure all of the calls tried to hit the cache before going to the database, reduced the load time of an average search that retrieved 200 services from about 90 seconds to less than 8 seconds.

-various fixes in Umbraco related to content creation based on the objects in our ERP product's backend

-integrated the DIBSpayment Easy API to our booking engine, in the angular application and in the backend ERP product

-added automatic image resizing and watermarking of uploaded images to our Umbraco CMS

-helped create and run various exports from our client's old system to our CMS

-had various calls with our client, talking about requirements, managing expectations, and deciding on the final UX.

-in another project, worked on a web scraper that kept the data from our client's platform in sync with ours


\subsection{Bitdefender}Cyber Threat Intelligence Lab | Junior Software Engineer | 01/06/2020 $\rightarrow$ 01/10/2020
\subsubsection{Technologies used: Node.js, Express.js, JQuery, Bootstrap, AJAX, JSON}
\subsubsection{Responsibilities:}
-refactoring an existing codebase for a log processing and analisys utility


-substantially optimizing the filtering and lowered general load times across the application

-implemented lazy loading in order to decrease the size of the requests sent and received

-added support for a different type of log file

-implemented automatic detection of log file type and OS


\section{Education}
\subsection{Currently getting a bachelor's degree at 
the Faculty of Computer Science Iasi(final year student)}
 8.5/10 Grade Average

\subsection{Winning project in FIIPractic full-stack .Net courses}
ASP.NET, .NET MVC Core


\end{document}




